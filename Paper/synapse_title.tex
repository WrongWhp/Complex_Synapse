
%%%%%%%%%%%%%%%%%%%%%%%%%%%%%%%%%%%%%%%%%%%%%%%%%%%%%%%%%%%%%%%%%%%%%%%%%%
\title{Optimal synaptic strategies for different timescales of memory}
%%%%%%%%%%%%%%%%%%%%%%%%%%%%%%%%%%%%%%%%%%%%%%%%%%%%%%%%%%%%%%%%%%%%%%%%%%

% Use letters for affiliations, numbers to show equal authorship (if applicable) and to indicate the corresponding author
\iftoggle{main} {\author[a,1]{Subhaneil Lahiri}
\author[a]{Surya Ganguli}}
{\author{Lahiri \textit{et al.}}}

% Please give the surname of the lead author for the running footer
\leadauthor{Lahiri}

\iftoggle{main}{%
\affil[a]{Department of Applied Physics, Stanford University, Stanford CA}
% Please add here a significance statement to explain the relevance of your work
\significancestatement{Authors must submit a 120-word maximum statement about the significance of their research paper written at a level understandable to an undergraduate educated scientist outside their field of speciality.}
% Please include corresponding author, author contribution and author declaration information
\authorcontributions{Please provide details of author contributions here.}
\authordeclaration{Please declare any conflict of interest here.}
%\equalauthors{\textsuperscript{1}A.O.(Author One) and A.T. (Author Two) contributed equally to this work (remove if not applicable).}
\correspondingauthor{\textsuperscript{1}To whom correspondence should be addressed. E-mail: sulahiri@stanford.edu}
%%%%%%%%%%%%%%%%%%%%%%%%%%%%%%%%%%%%%%%%%%%%%%%%%%%%%%%%%%%%%%%%%%%%%%%%%%
% Keywords are not mandatory, but authors are strongly encouraged to provide them. If provided, please include two to five keywords, separated by the pipe symbol, e.g:
\keywords{synaptic plasticity | learning | memory | neuroscience}
%\abbreviations{SNR, signal-to-noise ratio; LTP, long term potentiation;LTD, long term depression}
%%%%%%%%%%%%%%%%%%%%%%%%%%%%%%%%%%%%%%%%%%%%%%%%%%%%%%%%%%%%%%%%%%%%%%%%%%
\begin{abstract}
  An incredible gulf separates theoretical models of synapses, often described solely by a single scalar value denoting the size of a postsynaptic potential, from the immense complexity of molecular signaling pathways underlying real synapses.
  To understand the functional contribution of such molecular complexity to learning and memory, it is essential to expand our theoretical conception of a synapse from a single scalar to an entire dynamical system with many internal molecular functional states.
  Moreover, theoretical considerations alone demand such an expansion; network models with scalar synapses assuming finite numbers of distinguishable synaptic strengths have strikingly limited memory capacity.
  This raises the fundamental question, how does synaptic complexity give rise to memory?
  To address this, we develop new mathematical theorems elucidating the relationship between the structural organization and memory properties of complex synapses that are themselves molecular networks.
  Moreover, in proving such theorems, we uncover a framework, based on first passage time theory, to impose an order on the internal states of complex synaptic models, thereby simplifying the relationship between synaptic structure and function.

  Overall, we uncover general design principles governing the functional organization of complex molecular networks, and suggest new experimental observables in synaptic physiology, based on first passage time theory, that connect molecular complexity to memory.
\end{abstract}
}{}

\dates{This manuscript was compiled on \today}
\doi{\url{www.pnas.org/cgi/doi/10.1073/pnas.XXXXXXXXXX}}

