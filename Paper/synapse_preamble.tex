\iftoggle{main}{%
\usepackage{graphicx}
%\usepackage{graphics} for finding documentation only
\pdfsuppresswarningpagegroup=1
\usepackage{epstopdf}
\epstopdfsetup{update,suffix=-generated}
%
\IfFileExists{"C:/Progra\string~1/Inkscape/inkscape.exe"}{\newcommand{\inkscapedir}{C:/Progra\string~1/Inkscape/}}{}
\IfFileExists{"/Applications/Inkscape.app/Contents/Resources/bin/inkscape"}{\newcommand{\inkscapedir}{/Applications/Inkscape.app/Contents/Resources/bin/}}{}
\providecommand{\inkscapedir}{}
\epstopdfDeclareGraphicsRule{.svg}{pdf}{.pdf}{%
  "\inkscapedir inkscape" -f #1 -C -A \OutputFile}
%
\usepackage{adjustbox}
\newcommand{\aligntop}[1]{\adjustbox{valign=t}{#1}}
%\usepackage[inline]{enumitem}
\newlist{myenuma}{enumerate*}{4}
\setlist[myenuma]{label=(\alph*),ref=\alph*}
\newcommand{\panel}[1]{\item\aligntop{#1}}
\graphicspath{{"Figs/"}}
}{}%\iftoggle{main}
%%%%%%%%%%%%%%%%%%%%%%%%%%%%%%%%%%%%%%%%%%%%%%%%%%%%%%%%%%%%%%%%%%%%%%%%%%
\usepackage{amssymb,amsfonts}
\usepackage{amsthm}
%\usepackage[notref,notcite]{showkeys}
%\usepackage[centertags]{amsmath}
%\usepackage{hyperref}
\makeatletter
\renewcommand{\p@subsection}{\thesection.}
\renewcommand{\p@subsubsection}{\thesection.}
\makeatother
\usepackage{xr-hyper}
\newcommand{\main}{the main paper}
\newcommand{\supp}{supporting information}
\hypersetup{bookmarksopenlevel=1,bookmarksnumbered,extension=pdf,pdfstartview=FitH}
%--------------------------------------------------------------------------------
\usepackage[capitalise,nameinlink]{cleveref}
\Crefname{appendix}{Section}{Sections}
%--------------------------------------------------------------------------------
\makeatletter
\newcommand{\eq@pre}{[}
\newcommand{\eq@post}{]}
\crefformat{equation}{#2\cref@equation@name~\eq@pre#1\eq@post#3}
\crefrangeformat{equation}{\cref@equation@name@plural~\eq@pre#3#1#4--#5#2#6\eq@post}
\crefmultiformat{equation}%
    {\cref@equation@name@plural~\eq@pre#2#1#3}%
    {,#2#1#3\eq@post}%
    {,#2#1#3}%
    {,#2#1#3\eq@post}
\crefrangemultiformat{equation}%
    {\cref@equation@name@plural~\eq@pre#3#1#4--#5#2#6}%
    {,#3#1#4--#5#2#6\eq@post}%
    {,#3#1#4--#5#2#6}%
    {,#3#1#4--#5#2#6\eq@post}
\Crefformat{equation}{#2\Cref@equation@name~\eq@pre#1\eq@post#3}
\Crefrangeformat{equation}{\Cref@equation@name@plural~\eq@pre#3#1#4--#5#2#6\eq@post}
\Crefmultiformat{equation}%
    {\Cref@equation@name@plural~\eq@pre{#1}#2#1#3}%
    {,#2#1#3\eq@post}%
    {,#2#1#3}%
    {,#2#1#3\eq@post}
\Crefrangemultiformat{equation}%
    {\Cref@equation@name@plural~\eq@pre#3#1#4--#5#2#6}%
    {,#3#1#4--#5#2#6\eq@post}%
    {,#3#1#4--#5#2#6}%
    {,#3#1#4--#5#2#6\eq@post}
\makeatother
% ----------------------------------------------------------------
\newtheoremstyle{sldefinition}%
  {3pt}%space above
  {3pt}%space below
  {}%body font
  {}%indent amount
  {\bfseries}%theorem head font
  {}%theorem head punctuation
  {\newline}%space after head
  {\thmname{#1}\thmnumber{ #2}:{\bfseries\thmnote{ #3}}}%head spec
\newtheoremstyle{slplain}%
  {3pt}%space above
  {3pt}%space below
  {}%body font
  {}%indent amount
  {\bfseries}%theorem head font
  {}%theorem head punctuation
  {\newline}%space after head
  {\thmname{#1}\thmnumber{ #2}{\bfseries\thmnote{ (#3)}}: }%head spec
%
\theoremstyle{slplain}
\newtheorem{thm}{Theorem}
\newtheorem{cor}[thm]{Corollary}
\newtheorem{lem}[thm]{Lemma}
%
\theoremstyle{sldefinition}
\newtheorem{defn}{Definition}
%
\theoremstyle{remark}
\newtheorem*{notn}{Notation}
\newtheorem*{rem}{Remark}
% ----------------------------------------------------------------

%%%%%%%%%%%%%%%%%%%%%%%%%%%%%%%%%%%%%%%%%%%%%%%%%%%%%%%%%%%%%%%%%%%%%%%%%%
%% OPTIONAL MACRO DEFINITIONS
%%%%%%%%%%%%%%%%%%%%%%%%%%%%%%%%%%%%%%%%%%%%%%%%%%%%%%%%%%%%%%%%%%%%%%%%%%
\newcommand{\ra}{\rightarrow}
\newcommand{\lr}{\leftrightarrow}
\newcommand{\cdt}{\!\cdot\!}
\newcommand{\vp}{\vspace{0.5cm}}
\newcommand{\hp}{\hspace{0.5cm}}
%
%e.g., i.e. with normal spaces
\newcommand{\eg}{e.g.\ }
\newcommand{\ie}{i.e.\ }
\newcommand{\cf}{cf.\ }
\newcommand{\etc}{etc.\ }
\newcommand{\wrt}{wrt.\ }
\newcommand{\Eg}{E.g.\ }
\newcommand{\Ie}{I.e.\ }
\newcommand{\Cf}{Cf.\ }
\newcommand{\Etc}{Etc.\ }
%
\usepackage{delim}
% brackets etc.
\delimdef\prn#1{\dleft( #1 \dright)}
\delimdef\brc#1{\dleft\{ #1 \dright\}}
\delimdef\brk#1{\dleft[ #1 \dright]}
\delimdef\abs#1{\dleft\lvert #1 \dright\rvert}
\delimdef\nrm#1{\dleft\lVert #1 \dright\rVert}
\delimdef\av#1{\dleft\langle #1 \dright\rangle}
\delimdef\avc#1#2{\dleft\langle #1 \dmiddle\vert #2 \dright\rangle}
%\newenvironment{cases}{\dleft\{\begin{aligned}}{\end{aligned}\dright.}
%
% Sets
\delimdef\set#1#2{\dleft\{ #1 \dmiddle\vert #2 \dright\}}
\delimdef\cond#1#2{\!\dleft( #1 \dmiddle\vert #2 \dright)}
\delimdef\condb#1#2{\!\dleft[ #1 \dmiddle\vert #2 \dright]}
%
% Derivatives, etc. First argument is optional.
\newcommand{\diff}[3][]{\frac{\mathrm{d}^{#1} #2}{\mathrm{d}{#3}^{#1}}}
\newcommand{\pdiff}[3][]{\frac{\partial^{#1} #2}{\partial {#3}^{#1}}}
\newcommand{\pdiffc}[3][]{\left(\frac{\partial #2}{\partial {#3}}\right)_{\!\!#1}}
\newcommand{\intd}[2][]{\int #1\!\!\dr #2\,}
%
% Un-italicised letters
\newcommand{\dr}{\mathrm{d}}
\newcommand{\e}{\mathrm{e}}
\newcommand{\ir}{\mathrm{i}}
\DeclareMathOperator{\tr}{tr}
\DeclareMathOperator{\Tr}{Tr}
\DeclareMathOperator{\Det}{Det}
\DeclareMathOperator{\cov}{Cov}
\DeclareMathOperator{\var}{Var}
\DeclareMathOperator{\bias}{Bias}
\DeclareMathOperator{\rank}{rank}
\DeclareMathOperator{\sgn}{sgn}
\DeclareMathOperator{\sech}{sech}
\DeclareMathOperator{\diag}{diag}
%
% The default \Im and \Re look crap
\renewcommand{\Im}{\operatorname{\mathfrak{Im}}}
\renewcommand{\Re}{\operatorname{\mathfrak{Re}}}
%
% logic
\newcommand{\means}{\Longleftrightarrow}
\newcommand{\requires}{\Longleftarrow}
%
%additional symbols:
\newcommand{\half}{\frac{1}{2}}
\newcommand{\CO}{\mathcal{O}}
\newcommand{\CL}{\mathcal{L}}
\newcommand{\dt}{\dr t}
%
\DeclareMathOperator{\SNR}{SNR}
\DeclareMathOperator{\snr}{SNR}
\newcommand{\wv}{\vec{w}}
\newcommand{\wvi}{\vec{w}_\text{ideal}}
%matrices
\newcommand{\inv}{^{-1}}
\newcommand{\dg}{^\mathrm{dg}}
\newcommand{\trans}{^\mathrm{T}}
\newcommand{\I}{\mathbf{I}}
%vec of ones
\newcommand{\onev}{\mathbf{e}}
%mat of ones
\newcommand{\onem}{\mathbf{E}}
%Markov matrix
\newcommand{\MM}{\mathbf{Q}}
%prob distributions
\newcommand{\prob}{\mathbf{p}}
\newcommand{\eq}{\boldsymbol{\pi}}
%first passage times
\newcommand{\fpt}{\mathbf{T}}
%off-diag first passage times
\newcommand{\fptb}{\overline{\fpt}}
%fundamental matrix
\newcommand{\fund}{\mathbf{Z}}
%other symbols for matrices
\newcommand{\Pb}{\mathbf{P}}
\newcommand{\D}{\mathbf{D}}
\newcommand{\arow}{\boldsymbol{\xi}}
\newcommand{\Lb}{\boldsymbol{\Lambda}}
\newcommand{\w}{\mathbf{w}}
\newcommand{\W}{\mathbf{W}}
\newcommand{\M}{\mathbf{M}}
\newcommand{\enc}{\mathbf{q}}
\newcommand{\frg}{\W^{\mathrm{F}}}
\newcommand{\F}{\boldsymbol{\Phi}}
%superscripts
\newcommand{\pot}{^{\text{pot}}}
\newcommand{\dep}{^{\text{dep}}}
\newcommand{\potdep}{^{\text{pot/dep}}}
%sets
\newcommand{\CS}{\mathcal{S}}
\newcommand{\CA}{\mathcal{A}}
\newcommand{\CB}{\mathcal{B}}
\newcommand{\comp}{^\mathrm{c}}
%eigenmodes
\newcommand{\uv}{\mathbf{u}}
\newcommand{\vv}{\mathbf{v}}
\newcommand{\CI}{\mathcal{I}}
%%%%%%%%%%%%%%%%%%%%%%%%%%%%%%%%%%%%%%%%%%%%%%%%%%%%%%%%%%%%%%%%%%%%%%%%%%
