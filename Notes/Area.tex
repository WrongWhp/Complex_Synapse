% -*- TeX -*- -*- UK -*-
% ----------------------------------------------------------------
% arXiv Paper ************************************************
%
% Subhaneil Lahiri's template
%
% Before submitting:
%    Comment out hyperref
%    Comment out showkeys
%    Replace \input{mydefs.tex} with its contents
%       or include mydefs.tex in zip/tar file
%    Replace \input{newsymb.tex} with its contents
%       or include newsymb.tex in zip/tar file
%    Put this file, the .bbl file, any picture or
%       other additional files and natbib.sty
%       file in a zip/tar file
%
% **** -----------------------------------------------------------
\documentclass[12pt]{article}
% Preamble:
\usepackage{a4wide}
\usepackage[centertags]{amsmath}
\usepackage{amssymb}
%\usepackage{amsthm}
\usepackage[sort&compress,numbers]{natbib}
%\usepackage{citeB}
\usepackage{ifpdf}
%\usepackage{graphicx}
%\usepackage{graphics} for finding documentation only
%\usepackage{xcolor}
%\usepackage{pgf}
\ifpdf
\usepackage[pdftex,bookmarks]{hyperref}
\else
\usepackage[hypertex]{hyperref}
%\DeclareGraphicsRule{.png}{eps}{.bb}{}
\fi
%
% >> Only for drafts! <<
\usepackage[notref,notcite]{showkeys}
% ----------------------------------------------------------------
\vfuzz2pt % Don't report over-full v-boxes if over-edge is small
\hfuzz2pt % Don't report over-full h-boxes if over-edge is small
%\numberwithin{equation}{section}
%\renewcommand{\baselinestretch}{1.5}
% ----------------------------------------------------------------
% New commands etc.
\input{mydefs.tex}
\input{newsymb.tex}

\newcommand{\means}{\Longleftrightarrow}
\newcommand{\requires}{\Longleftarrow}
\newcommand{\I}{\mathbf{I}}
\newcommand{\M}{\mathbf{M}}
\newcommand{\pp}{\mathbf{p}^\infty}
\newcommand{\T}{\mathbf{T}}
\newcommand{\Zb}{\mathbf{Z}}
\newcommand{\w}{\mathbf{w}}
\newcommand{\onev}{\mathbf{c}}
\newcommand{\pib}{\boldsymbol{\pi}}
%
%%%%%%%%%%%%%%%%%%%%%%%%%%%%%%%%%%%%%%%%%%%%%%%%%%%%%%%%%%%%%%%%%%%%%%%%%%
% Title info:
\title{Area maximisation}
%
% Author List:
%
\author{Subhaneil Lahiri
%
}

\begin{document}

\maketitle



%%%%%%%%%%%%%%%%%%%%%%%%%%%%%%%%%%%%%%%%%%%%%%%%%%%%%%%%%%%%%%%%%%%%%%%%%%


\begin{abstract}
  We extremise the area under the synaptic memory curve
\end{abstract}


%%%%%%%%%%%%%%%%%%%%%%%%%%%%%%%%%%%%%%%%%%%%%%%%%%%%%%%%%%%%%%%%%%%%%%%%%%
% Beginning of Article:
%%%%%%%%%%%%%%%%%%%%%%%%%%%%%%%%%%%%%%%%%%%%%%%%%%%%%%%%%%%%%%%%%%%%%%%%%%

It will be convenient for us to define the diagonal elements of $\T$ to be zero (it is conventional to define them to be the recurrence times, $\T_{ii} = 1/\pp_i$, but that would lead to a lot of extra $\delta_{ik}$'s for us). Similarly, we define $N^k_{ij}=0$ if $i=k$ or $j=k$.


\section{Mixing times}\label{sec:mixing}

Define
%
\begin{equation}\label{eq:mixingpm}
  \eta^\pm_i \equiv \sum_k \T_{ik} \pp_k \prn{\frac{1 \pm \w_k}{2}}.
\end{equation}
%
It can be shown that \cite{hunter2006mixing}\footnote{Not that their $\eta$ is equal to our $\eta+1$ due to the difference in diagonal elements of $\T$.}
%
\begin{equation}\label{eq:mixing}
  \eta^+_i + \eta^-_i = \eta \qquad (\text{independent of }i).
\end{equation}
%

It would be nice if we could prove (at least for the maximal area) that
%
\begin{equation}\label{eq:orderinghelp}
  \w_i = +1 \quad \text{and} \quad \w_j = -1
  \qquad \implies \qquad
  \eta^+_i < \eta^+_j
  \quad \means \quad
  \eta^-_i > \eta^-_j
\end{equation}
%
then we could use these quantities to order the states.

The area under the memory curve is given by
%
\begin{equation}\label{eq:areamixing}
\begin{aligned}
  A &= \sqrt{N}(4f^+f^-) \sum_{i \neq j} q_{ij} \pp_i (\eta^+_i - \eta^+_j)
   &&= -\sqrt{N}(4f^+f^-) \sum_{i , j} q_{ij} \pp_i \eta^+_j\\
   &= \sqrt{N}(4f^+f^-) \sum_{i \neq j} q_{ij} \pp_i (\eta^-_j - \eta^-_i)
   &&= \sqrt{N}(4f^+f^-) \sum_{i , j} q_{ij} \pp_i\eta^-_j .
\end{aligned}
\end{equation}
%

\section{Shifting $q_{ij}$}\label{sec:shiftq}

Consider the following change
%
\begin{equation}\label{eq:shiftq}
  \M^+_{ij} \to \M^+_{ij} + f^-\epsilon_{ij},
  \qquad
  \M^-_{ij} \to \M^-_{ij} - f^+\epsilon_{ij},
  \qquad
  \sum_j \epsilon_{ij} = 0.
\end{equation}
%
This leaves $\M$ unchanged, and therefore $\pp$, $\T$, $\eta$, $N^j_{ik}$ and $H^j_{ik}$ as well. This means
%
\begin{equation}\label{eq:shiftqarea}
  A \to A + \sqrt{N}(4f^+f^-) \sum_{i \neq j} \epsilon_{ij} \pp_i (\eta^+_i - \eta^+_j).
\end{equation}
%

Suppose $\eta^+_i > \eta^+_j$. We can increase $A$ by making $\epsilon_{ij}>0$. The only thing that could stop us is if $\M^-_{ij}$ or $\M^+_{ii}$ hits zero (which also takes care of the possibility that $\M^+_{ij}$ hits unity). Similar considerations for $\eta^+_i < \eta^+_j$ imply that at the maximum:
%
\begin{equation}\label{eq:shiftqmax}
  \begin{aligned}
    \eta^+_i &> \eta^+_j \qquad\implies\qquad & \M^-_{ij} &= 0 \quad&\text{or}\quad \M^+_{ii} &= 0 ,\\
    \eta^+_i &< \eta^+_j \qquad\implies & \M^+_{ij} &= 0 &\text{or}\quad \M^-_{ii} &= 0 .
  \end{aligned}
\end{equation}
%
If $\eta^+_i = \eta^+_j$, $\epsilon_{ij}$ is a null direction, so we can impose either of the two conditions if we wish. If it weren't for the second possibility in each case, this would imply an upper/lower triangular structure for $\M^\pm$.

If we have a trio $\eta^+_i > \eta^+_j > \eta^+_k$, two of the transitions can help each other: decreasing $\M^-_{jk}$ increases $\M^-_{jj}$, which allows us to further decrease $\M^+_{ji}$. Decreasing $\M^+_{ji}$ increases $\M^+_{jj}$, which allows us to further decrease $\M^-_{jk}$.
Successful triangulation requires:
%
\begin{equation}\label{eq:trianglesuccess}
  \begin{aligned}
    \sum_{\{i \mid \eta^+_i > \eta^+_j\}}\!\!\!\!\! \M^+_{ji}
     &\leq \M^-_{jj} + \!\!\!\!\!\sum_{\{i \mid \eta^+_i < \eta^+_j\}}\!\!\!\!\! \M^-_{ji}
    \qquad \implies&
    \sum_{\{i \mid \eta^+_i > \eta^+_j\}}\!\!\!\!\! \prn{\M^+_{ji} + \M^-_{ji}}
     \leq 1 ,\\
    \sum_{\{i \mid \eta^+_i < \eta^+_j\}}\!\!\!\!\! \M^-_{ji}
     &\leq \M^+_{jj} + \!\!\!\!\!\sum_{\{i \mid \eta^+_i > \eta^+_j\}}\!\!\!\!\! \M^+_{ji}
    \qquad \implies&
    \sum_{\{i \mid \eta^+_i < \eta^+_j\}}\!\!\!\!\! \prn{\M^+_{ji} + \M^-_{ji}}
     \leq 1 .\\
  \end{aligned}
\end{equation}
%
The two left-hand-sides of the final inequalities sum to $(2-\M^+_{jj}-\M^-_{jj})$, so they are not quite inconsistent.


%Guaranteeing that this process will allow us to set $\M^-_{jk}=\M^+_{ji}=0$ would probably require some sort of symmetry between potentiation and depression.

\section{Generalised fundamental matrix}\label{sec:fundamental}

Define the generalised fundamental matrix as
%
\begin{equation}\label{eq:funddef}
  \Zb = (\I - \M + \onev\pib)^{-1},
\end{equation}
%
where $\onev$ is a row vector of 1's and $\pib$ is any row vector such that $\pib\onev=1$. The usual fundamental matrix has $\pib=\pp$ \cite[\S3.2]{kemeny1960finite}, but it is more convenient to let it be independent of $\M$ \cite[App.VIII]{kemeny1960finite}, \eg $\pib=\onev^\mathrm{T}/n$.

Then we have
%
\begin{equation}\label{eq:fromfund}
  \begin{aligned}
    \pp &= \pib \Zb, \\
    \Zb \onev &= \onev , \\
    \T_{ij} &= \frac{\Zb_{jj}-\Zb_{ij}}{\pp_j}, \\
    \eta^+_i - \eta^+_j &= \half\sum_k (\Zb_{jk}-\Zb_{ik})\w_k.
  \end{aligned}
\end{equation}
%

This allows us to write the area as
%
\begin{equation}\label{eq:arefund}
  A = \sqrt{N}(2f^+f^-)\sum_{ijkl} q_{ij} \pib_l \Zb_{li} (\Zb_{jk} - \Zb_{ik}) \w_k.
\end{equation}
%


\section{Derivatives wrt.\ $\M^\pm_{ij}$}\label{sec:derivs}

We will regard the off-diagonal elements of $\M^\pm_{ij}$ to be the independent variables, with $\M^\pm_{ii}=1-\sum_{j \neq i} M^\pm_{ij}$. Thus,
%
\begin{equation}\label{eq:basicderivs}
%  \pdiff{\M_{ij}}{\M_{gh}} = \delta_{gi}(\delta_{hj}-\delta_{ij}),
%  \qquad
%  \pdiff{f(\M)}{\M^\pm_{gh}} = f^\pm \pdiff{f(\M)}{\M_{gh}},
  \pdiff{\M_{ij}}{\M^\pm_{gh}} = f^\pm \delta_{gi}(\delta_{hj}-\delta_{ij}),
  \qquad
  \pdiff{q_{ij}}{\M^\pm_{gh}} = \pm\delta_{gi}(\delta_{hj}-\delta_{ij}).
\end{equation}
%
The implicit $g \neq h$ that comes with all derivatives is unnecessary, as the derivatives above vanish when $g=h$.

Differentiating \eqref{eq:funddef},
%
\begin{equation}\label{eq:fundderiv}
  \pdiff{\Zb_{ij}}{\M_{gh}} = \Zb_{ig}(\Zb_{hj}-\Zb_{gj}).
\end{equation}
%
From \cite{cho2000markov}, we have
%
\begin{equation}\label{eq:pderiv}
  \pdiff{\pp_k}{\M_{gh}} = \pp_k \pp_g (\T_{gk} - \T_{hk}).
\end{equation}
%

We can write \cite{grinstead1997introduction}
%
\begin{equation}\label{eq:NTexpr}
  \M^{(k)}_{ij} = (1-\delta_{ik})(1-\delta_{jk}) \M_{ij},
  \qquad
  N^k_{ij} = (1-\delta_{jk})(I-\M^{(k)})^{-1}_{ij},
  \qquad
  \T_{ik} = \sum_{j} N^k_{ij},
\end{equation}
%
with $i,j \neq k$. Differentiating,
%
\begin{equation}\label{eq:NTderiv}
\begin{aligned}
  \pdiff{N^k_{ij}}{\M_{gh}} &= -N^k_{ig}[ N^k_{gj} - N^k_{hj}],\\
  \pdiff{\T_{ik}}{\M_{gh}} &= -N^k_{ig}[\T_{gk} - \T_{hk}],
\end{aligned}
\end{equation}
%
again with $i,j \neq k$.


\section{Kuhn-Tucker conditions}

Consider the Lagrangian
%
\begin{equation}\label{eq:lagrangian}
  \CL = \frac{A}{\sqrt{N}(2f^+f^-)} + \sum_{\pm}\sum_{ij} \mu_{ij} \M^\pm_{ij}.
\end{equation}
%
Necessary conditions for an extremum are
%
\begin{equation}\label{eq:extremum}
  \pdiff{\CL}{\M^\pm_{gh}} = 0,
  \qquad
    \mu^\pm_{ij} \geq 0,\quad
    \M^\pm_{ij} \geq 0,\quad
    \mu^\pm_{ij}\M^\pm_{ij} = 0.
\end{equation}
%
with $g \neq h$. We will make statements of the following form repeatedly:
%
\begin{equation}\label{eq:ineqstatement}
  A < B \qquad \implies \qquad
  A < 0 \quad \text{or}\quad B > 0.
\end{equation}
%

Let $\lambda^\pm_{ij} = \mu^\pm_{ij} - \mu^\pm_{ii}$. Together with the bounds \eqref{eq:extremum}, applying \eqref{eq:ineqstatement} leads to
%
\begin{equation}\label{eq:lambdaineqs}
  \begin{aligned}
    \lambda^\pm_{ij}  &> 0
    \qquad &\implies \qquad
    \mu^\pm_{ij} > 0
    \qquad &\implies \qquad
    \M^\pm_{ij} = 0, \\
    \lambda^\pm_{ij}  &< 0
    \qquad &\implies \qquad
    \mu^\pm_{ii} > 0
    \qquad &\implies \qquad
    \M^\pm_{ii} = 0. \\
  \end{aligned}
\end{equation}
%
Suppose $\M^+_{ii}=0$, then for some $j$, $\M^+_{ij}\neq0$ and that $\lambda^+_{ij}<0$. We can see that
%
\begin{equation}\label{eq:lambdadiagineq}
  \M^\pm_{ii}=0 \quad\means\quad \min_j \lambda^\pm_{ij} <0.
\end{equation}
%


Using the formulae of \S\ref{sec:derivs},
%
\begin{equation}\label{eq:lagrangderiv}
  \begin{aligned}
    \pdiff{\CL}{\M^\pm_{gh}} =&\, \lambda^\pm_{gh}
    \pm \sum_k \pp_g \pp_k \w_k (\T_{gk} - \T_{hk}) \\
      &+ f^\pm \sum_k \sum_{i \neq j} q_{ij} \pp_i \pp_k \w_k \pp_g
         (\T_{ik} - \T_{jk}) (\T_{gi} - \T_{hi} + \T_{gk} - \T_{hk})\\
      &- f^\pm \sum_k \sum_{i \neq j} q_{ij} \pp_i \pp_k \w_k
         (N^k_{ig} - N^k_{jg}) (\T_{gk} - \T_{hk})\\
    =&\, \lambda^\pm_{gh}
    \pm \sum_{kl} \w_k \pib_l \Zb_{lg} (\Zb_{hk} - \Zb_{gk}) \\
      &+ f^\pm \sum_{ijkl} q_{ij} \w_k \pib_l \brk{
         \Zb_{lg}(\Zb_{hi}-\Zb_{gi})(\Zb_{jk}-\Zb_{ik}) + \Zb_{li}(\Zb_{jg}-\Zb_{ig})(\Zb_{hk}-\Zb_{gk}) } \\
    =&\, \lambda^\pm_{gh}
    \pm 2 \pp_g (\eta^+_g - \eta^+_h) \\
      &+ 2f^\pm \sum_{ij} q_{ij} \pp_i \pp_g \brk{
         (\T_{gi}-\T_{hi})(\eta^+_{i}-\eta^+_{j}) +
         (\T_{ig}-\T_{jg})(\eta^+_{g}-\eta^+_{h}) } \\
    =& \, 0.
  \end{aligned}
\end{equation}
%

Consider
%
\begin{equation}\label{eq:derivsshiftq}
  f^- \pdiff{\CL}{\M^+_{gh}} - f^+ \pdiff{\CL}{\M^-_{gh}}
   = (f^- \lambda^+_{gh} - f^+ \lambda^-_{gh}) + 2\pp_g (\eta^+_g - \eta^+_h)
   = 0,
\end{equation}
%
which corresponds to the shift considered in \S\ref{sec:shiftq}. We can see that
%
\begin{equation}\label{eq:etalambdaineq}
  \begin{aligned}
    \eta^+_g &> \eta^+_h
    &\quad & \implies \quad&
    f^+ \lambda^-_{gh} &> f^- \lambda^+_{gh}
    &\quad & \implies \quad&
    \lambda^-_{gh} &> 0
    &\quad & \text{or} \quad&
    \lambda^+_{gh} &< 0
    \\ &&&&&
    &\quad & \implies \quad&
    \M^-_{gh} &= 0
    &\quad & \text{or} \quad&
    \M^+_{gg} &= 0,
    \\
    \eta^+_g &< \eta^+_h
    &\quad & \implies \quad&
    f^- \lambda^+_{gh} &> f^+ \lambda^-_{gh}
    &\quad & \implies \quad&
    \lambda^+_{gh} &> 0
    &\quad & \text{or} \quad&
    \lambda^-_{gh} &< 0
    \\ &&&&&
    &\quad & \implies \quad&
    \M^+_{gh} &= 0
    &\quad & \text{or} \quad&
    \M^-_{gg} &= 0,
  \end{aligned}
\end{equation}
%
which we've already seen in \S\ref{sec:shiftq}.

Consider
%
\begin{equation}\label{eq:derivsoverlap}
 \begin{aligned}
   \pdiff{\CL}{\M^+_{gh}} + \pdiff{\CL}{\M^-_{gh}}
   =&\, (\lambda^+_{gh} + \lambda^-_{gh})
       +  2 \sum_{ij} q_{ij} \pp_i \pp_g \brk{
         (\T_{gi}-\T_{hi})(\eta^+_{i}-\eta^+_{j}) +
         (\T_{ig}-\T_{jg})(\eta^+_{g}-\eta^+_{h}) }\\
    =& \, 0.
 \end{aligned}
\end{equation}
%
If we could argue that $\lambda^+_{gh} + \lambda^-_{gh} > 0$, this would imply that $\M^+_{gh}\M^-_{gh}=0$.

%\section*{Acknowledgements}



%%%%%%%%%%%%%%%%%%%%%%%%%%%%%%%%%%%%%%%%%%%%%%%%%%%%%%%%%%%%%%%%%%%%%%%%%%
%\section*{Appendices}
%\appendix
%%%%%%%%%%%%%%%%%%%%%%%%%%%%%%%%%%%%%%%%%%%%%%%%%%%%%%%%%%%%%%%%%%%%%%%%%%





%%%%%%%%%%%%%%%%%%%%%%%%%%%%%%%%%%%%%%%%%%%%%%%%%%%%%%%%%%%%%%%%%%%%%%%%%%

\bibliographystyle{utcaps_sl}
\bibliography{maths}

\end{document}
